% global TikZ styles
\tikzset{
    every node/.style = {draw = none,
	                     inner sep = 0pt,
	                     outer sep = 0pt},
    %
    grid/.style={draw,step=#1,gray,very thin,dotted},
    grid/.default={1.0cm},
    %
    axis/.style={draw,densely dotted,gray,font=\small,text=gray,->,>=stealth},
    %
    fixture/.style 2 args = {fill,
                             pattern = north east lines,
                             minimum width  = #1*1cm,
                             minimum height = #2*1cm},
    %
    spring/.style = {draw,
                     decorate,decoration = {coil,
                                            aspect         = 0.5,
                                            segment length = #1*1mm,
                                            amplitude      = 2mm,
                                           }
                },
    spring/.default = {1},
    %
    mass/.style = {rectangle,
                   minimum size=#1*1cm,
                   draw=black,
                   fill=lightgray,
                   very thin},
    mass/.default={0.7},
    %
    force/.style={->,
                  >=latex,
                  draw=blue,
                  fill=blue,
                  thick},
}  % \tikzset

\newcommand{\VerticalMassSpringSystem}[5]
    % Vertical Mass Spring System
    % #1 : length         of spring's coil
    % #2 : stretch factor of spring's coil
    % #3 : size of square representing mass    <== must be a non-negative integer
    %                                              0 means no mass is drawn
    % #4 : label for spring's extension
	% #5 : horizontal shift of sub figure
    % !!!   No terminating ';' when calling this command   !!!
{
    \begin{scope}[xshift=#5]
        % define constants
        \def\LENGTH{2}              % length of fixture
        \def\THICKNESS{0.25}        % thickness of fixture
        \def\ATTACHMENT{0.50}       % length of spring's attachment
        \def\SPRING{#1}             % relaxed length of spring's coil
        \def\STRETCH{#2}            % stretch factor of spring's coil
        \def\MASS{#3}               % radius of circle representing mass

        % define coordinates
		\path coordinate (FM)  at (0,0);                                % mid point of fixture
		\path coordinate (FL)  at ( $(FM) + (-0.5*\LENGTH,0)$ );        % left edge of fixture
		\path coordinate (FR)  at ( $(FM) + (+0.5*\LENGTH,0)$ );        % right edge of fixture

		\path coordinate (T) at (0,-\ATTACHMENT);                       % top of spring's coil
		\path coordinate (B) at (0,-\ATTACHMENT-\SPRING*\STRETCH);      % bottom of spring's coil
        \path coordinate (R) at (0,-\ATTACHMENT-\SPRING-\ATTACHMENT);   % free end of relaxed spring

		\path coordinate (MT) at ( $(B)  - (0,\ATTACHMENT)$ );          % top of mass
		\path coordinate (MC) at ( $(MT) - (0,+0.5*\MASS)$ );           % center of mass

        % draw fixture
		\path[draw,very thick] (FL) -- (FR);
		\path (FM) node[fixture={\LENGTH}{\THICKNESS}, anchor = south] {};

		% connect spring to fixture
		\path[draw] (FM) -- (T);

		% draw spring's coil
		\path[spring=\STRETCH] (T) -- (B);

		% connect spring to mass
		\path[draw] (B) -- (MT);

        \ifthenelse{\MASS > 0}
        {
    		% draw mass
            \path (MC) node[mass=\MASS] (M) {$m$};
        }

        % draw coordinate system
        \draw[axis] (R) -- +(-0.9,0) -- +(+0.9,0) node[below right] {$x$};
        \draw[axis] (R) -- +(0,-3.5) node[below right] {$y$};

		% indicate spring's relaxed length
		\draw[dashed] (FM)  +(-0.1,0)     --  +(-0.9,0);
        \draw[dashed] (R) +(-0.1,0)     --  +(-0.9,0);
		\draw[latex-latex,dashed] ( $(FM) + (-0.5, 0.0)$ ) -- node[fill=white] {$l_0$} ( $(R) + (-0.5, 0.0)$ );
        \ifthenelse{\MASS > 0}
        {
            % indicate spring's extension
            \draw[dashed] (R) +(-0.1,0)     --  +(-0.9,0);
            \draw[dashed] (MT) +(-0.1,0)     --  +(-0.9,0);
            \draw[latex-latex,dashed] ( $(R) + (-0.5, 0.0)$ ) -- node[fill=white] {#4} ( $(MT) + (-0.5, 0.0)$ );
        }

    \end{scope}  % [xshift=#5]
}  % \newcommand{\VerticalMassSpringSystem}


\newcommand{\HorizontalMassSpringSystem}[4]
    % horizontal harmonic oscillator
    % #1 : length         of spring's coil
    % #2 : stretch factor of spring's coil
    % #3 : size of square representing mass
    % #4 : vertical offset of sub figure
    % !!!   No terminating ';' when calling this command   !!!
{
    \begin{scope}[yshift=#4]
        % define constants
        \def\LENGTH{2*#1}    		% length    of fixture's floor
        \def\HEIGHT{#3}    			% height    of fixture's wall
        \def\THICKNESS{0.25}    	% thickness of fixture
        \def\ATTACHMENT{0.5}		% length of spring's attachment
        \def\SPRING{#1}				% length of relaxed spring
        \def\STRETCH{#2}			% stretch factor of spring's coil
        \def\MASS{#3}				% size of square representing mass

        % define coordinates
        \path coordinate (E)  at   (0,0);                    			             % edge of fixture
        \path coordinate (F)  at ( $(E) + (0,0.5*\HEIGHT)$ );                        % fix point of spring        
        \path coordinate (L)  at ( $(F) + (\ATTACHMENT,0)$ );                        % left  point of spring's coil
        \path coordinate (R)  at ( $(L) + (\SPRING*\STRETCH,0)$ );                   % right point of spring's coil
        \path coordinate (M)  at ( $(R) + (\ATTACHMENT+0.5*\MASS,0)$ );	             % center of mass
        \path coordinate (EQ) at ( $(F) + (2*\ATTACHMENT+\SPRING+0.5*\MASS,0)$ );    % mass' equilibrium position

        % draw floor
        \path[draw,very thick] (E) -- (0:\LENGTH);
        \path (E) node[fixture={\LENGTH}{\THICKNESS}, anchor=north west] {};

        % draw left wall
        \path[draw,very thick] (E) -- (90:\HEIGHT);
        \path (E) node[fixture={\THICKNESS}{\HEIGHT}, anchor=south east] {};
        \path (E) node[fixture={\THICKNESS}{\THICKNESS}, anchor=north east] {};

        % draw connection of spring to wall
        \path[draw] (F) -- (L);

        % draw spring's coil
        \path[spring=\STRETCH] (L) -- (R);

        % draw connection of spring to mass
        \path[draw] (R) -- ( $(M)-(0.5*\MASS,0)$ );

        % draw mass
        \path (M) node[mass=\MASS,yshift=0.4pt] (M) {$m$};

        % draw coordinate system
        \draw[axis] (EQ) -- +(-2,0) -- +(2,0) node[below right] {$x$};
        \draw[axis] (EQ) -- +(0,1) node[above left] {$y$};
    \end{scope}  % [yshift=#4]
}  % \newcommand{\HorizontalMassSpringSystem}

